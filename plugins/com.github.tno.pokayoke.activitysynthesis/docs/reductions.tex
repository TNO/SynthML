% !TEX TS-program = pdflatex
% !TEX encoding = UTF-8 Unicode

% This is a simple template for a LaTeX document using the "article" class.
% See "book", "report", "letter" for other types of document.

\documentclass[11pt]{article} % use larger type; default would be 10pt

\usepackage[utf8]{inputenc} % set input encoding (not needed with XeLaTeX)
\usepackage{graphicx} % support the \includegraphics command and options

%%% PACKAGES
\usepackage{booktabs} % for much better looking tables
\usepackage{array} % for better arrays (eg matrices) in maths
\usepackage{paralist} % very flexible & customisable lists (eg. enumerate/itemize, etc.)
\usepackage{verbatim} % adds environment for commenting out blocks of text & for better verbatim
\usepackage{subfig} % make it possible to include more than one captioned figure/table in a single float
\usepackage{amsmath}
\usepackage{amsfonts}

%%% SECTION TITLE APPEARANCE
\usepackage{sectsty}
\allsectionsfont{\sffamily\mdseries\upshape} % (See the fntguide.pdf for font help)
% (This matches ConTeXt defaults)

%%% ToC (table of contents) APPEARANCE
\usepackage[nottoc,notlof,notlot]{tocbibind} % Put the bibliography in the ToC
\usepackage[titles,subfigure]{tocloft} % Alter the style of the Table of Contents
\renewcommand{\cftsecfont}{\rmfamily\mdseries\upshape}
\renewcommand{\cftsecpagefont}{\rmfamily\mdseries\upshape} % No bold!

%%% Custom commands
\newcommand{\andlabel}[0]{\textsf{and}}
\newcommand{\Bool}[0]{\mathbb{B}}
\newcommand{\Config}[0]{\textsf{Config}}
\newcommand{\coreedge}[0]{\varepsilon}
\newcommand{\Effects}[0]{\textsf{Effects}}
\newcommand{\Guard}[0]{\textsf{Guard}}
\newcommand{\orlabel}[0]{\textsf{or}}
\newcommand{\outgoing}[0]{\textsf{out}}
\newcommand{\outgoingguard}[0]{\textsf{guard}_{\outgoing}}
\newcommand{\State}[0]{\textsf{State}}

\title{Reduction Rules for Poka Yoke Activities}
%\author{}
%\date{} % Activate to display a given date or no date (if empty),
         % otherwise the current date is printed 

\begin{document}
\maketitle

\section{Introduction}
\label{sec:intro}

This document defines reduction rules for (simple UML-like) activities. 
These activities and their semantics are slightly different from other UML semantics, e.g. fUML, activities. 
For example, every action has an associated guard and effects that specify how action execution influences the system state. 
Edges have two guards, named incoming and outgoing guard, that define when a token can be placed on an edge, and when it can be removed, respectively. 
This document aims at finding patterns of action, nodes, edges that can be transformed into simpler patterns, while maintaining the same operational semantic of the activity. 
We call these transformations \emph{reductions}. 

%%%%%%%%%%%%%%%%%%%%%%%%%%%%%%%%%%%%%%%%%%%%%%%%%%%%%%%%%%%%%%

\section{Preliminaries and Notation}
\label{sec:prelim}

We largely refer to the Operation Semantics of Poka Yoke Activities for the static structure and dynamics behaviors of activities. 
We recall here only the elements that are useful for this document. 

\subsection{Statics}
\label{subsec:statics}

\paragraph{State.}
Activities are defined in the presence of \emph{state}, for example a set of variables and their current valuation.
We keep the notion of state more abstract for the purpose of defining the semantics, and let $\State$ be the set of all states.
Users of this semantics could later instantiate $\State$ as desired, e.g., as variable valuation mappings.
Let $\sigma \in \State$ be a typical state.

\paragraph{Guards and effects.}
Guards are defined to be state predicates, while effects are defined to be state transformers, i.e., functions that map states to any number of `successor states', to be able to model nondeterministic action behavior.

Let $\Guard = \State \rightarrow \Bool$ be the set of all \emph{guards} and $\Effects = \State \rightarrow 2^{\State}$ be the set of all \emph{effects functions} over states, with $2^{\State}$ the powerset of $\State$.
We use $g \in \Guard$ and $u \in \Effects$ to denote a typical guard and effects function, respectively.

\subsection{Dynamics}
\label{subsec:dynamics}

This section defines the dynamic behavior of core activities, by means of a token passing style operational semantics.
Its rules describe how to go from one configuration to another, where a configuration is essentially a set of edges holding a token together with a `current' state.
There are two token-passing rules: one for $\andlabel$-style and one for $\orlabel$-style action execution.

\paragraph{Configurations.}
The semantics of core activities is essentially defined as a relation between \emph{configurations}, in the sense that an execution step in an activity gets you from one configuration to another configuration.
A configuration describes which edges currently hold a token, and what the current state is.

More formally, let $\Config = 2^\mathcal{E} \times \State$ be the set of all configurations.
A configuration $(\Sigma, \sigma) \in \Config$ is a pair with $\Sigma \subseteq \mathcal{E}$ a set of edges---the ones currently holding a token---and $\sigma$ a `current' state.
We write $c \in \Config$ to denote a typical configuration, such that $c = (\Sigma, \sigma)$.

Any edge $\coreedge$ is said to be \emph{enabled in $c$} if $\coreedge \in \Sigma$ and $\outgoingguard(\coreedge)(\sigma)$ holds, i.e., if $\coreedge$ has a token in $c$ and if the outgoing guard of $\coreedge$ holds in $c$.
If $c$ is clear from the context, we may just say that $\coreedge$ is \emph{enabled}.

%%%%%%%%%%%%%%%%%%%%%%%%%%%%%%%%%%%%%%%%%%%%%%%%%%%%%%%%%%%%%%

\section{Reductions}
\label{sec:reductions}

\subsection{Series Edge Merging}



\end{document}
